%
% PROJECT: A Broadbased RF spectrum occupancy comparison of Blacksburg and Chicago
%   TITLE: LaTeX report template for ETDs in LaTeX
%  AUTHOR: Ramakrishnan Kalyanaraman, rk126@vt.edu
%     URL: http://etd.vt.edu/latex/
% SAVE AS: MEng_Report.tex
% REVISED: December 18, 2014 [GMc 8/30/10]
% 

% Instructions: Remove the data from this document and replace it with your own,
% keeping the style and formatting information intact.  More instructions
% appear on the Web site listed above.

\documentclass[12pt,sts]{report}

\setlength{\textwidth}{6.5in}
\setlength{\textheight}{8.5in}
\setlength{\evensidemargin}{0in}
\setlength{\oddsidemargin}{0in}
\setlength{\topmargin}{0in}

\setlength{\parindent}{0pt}
\setlength{\parskip}{0.1in}

\usepackage{enumerate}

% Uncomment for double-spaced document.
% \renewcommand{\baselinestretch}{2}

\usepackage{setspace}
\usepackage{url}
\usepackage{float}

\usepackage[english]{babel}
\usepackage{graphicx}
\usepackage{listings}
% \floatstyle{boxed}
% \restylefloat{figure}

\doublespacing
% \usepackage{epsf}
\setcounter{tocdepth}{5}
\begin{document}

\bibliographystyle{plain}
\thispagestyle{empty}
\pagenumbering{roman}
\begin{center}

% TITLE
{\Large 
Broad based RF Spectrum Occupancy survey of Blacksburg, VA and Chicago, IL
}

\vfill

\textbf{Ramakrishnan Kalyanaraman} \\
Master of Engineering \\
in \\
Computer Engineering \\
Bradley Department of Electrical and Computer Engineering \\
Virginia Polytechnic Institute and State University \\
Email: rk126@vt.edu

\vfill

\textbf{Committee members} \\
Dr. Allen B. MacKenzie \\
Dr. Luiz A. DaSilva \\
Dr. Jeffrey H. Reed

\vfill

October 18, 2014 \\
Blacksburg, Virginia

\end{center}

\pagebreak

\thispagestyle{empty}

\begin{center}

\textbf{ABSTRACT}

\end{center}

In this report we will briefly discuss a broad-based RF spectrum survey conducted at Virginia Tech, Blacksburg, VA. The measurements for this spectrum study were collected from Virginia Tech's Spectrum Observatory whose installation is also discussed in this report. We compare the occupancy measurement plots for arbitrary weekday(s), regular weekend and college football (game day) weekends, looking for consistently varying occupancy patterns. We report only those bands which particularly shows variations in primary user occupancy over these periods. We also carry a similar comparison of RF spectrum of Blacksburg, VA and Chicago, IL on an arbitrarily chosen busy weekday.

\vfill

\pagebreak

% Dedication and Acknowledgments are both optional
% \chapter*{Dedication}
% \chapter*{Acknowledgments}

\tableofcontents

\pagebreak

\listoffigures
\pagebreak

\listoftables
\pagebreak

\section*{Acknowledgments}

I am immensely grateful to my advisory chair, Dr Allen B. MacKenzie and my committee members, Dr Luiz A DaSilva and Dr Jeffrey H Reed who have provided excellent support and guidance, thereby encouraging me throughout my graduate school.

I would extend like to extend my acknowledgment to our project partners in Finland and Chicago for providing information on setting up the Virginia Tech Spectrum Observatory. I would like to thank my lab members Mr Abdallah S Abdallah and Mr Cameron Patterson for providing help in the installation of the observatory.

Finally, I thank my family and friends for their extended support.
\pagebreak

\pagenumbering{arabic}
\pagestyle{myheadings}

\renewcommand\thesection{\arabic{section}}

\section{Introduction}
% \setcounter{section}{1}

From the past two decades, wireless researchers have been conducting spectrum measurement campaigns, surveying spectrum occupancy in both urban and rural areas and also trying to come up with a mathematical model characterizing spectral usage. Most of their studies have shown that numerous spectrum bands lie vacant, though these bands are licensed by the regulatory bodies. Recently, Dynamic Spectrum Access (DSA) \cite{Akyildiz06nextgeneration/dynamic} has emerged as one of the plausible technique that would utilize the spectrum with maximum efficiency. So what is Dynamic Spectrum Access? With several bands (especially federal bands) that remain empty over a long period of time, while, there are bands which become heavily congested, DSA enables an opportunistic access of these sparsely occupied bands to the secondary users. These secondary users are issued with a license to use these empty bands and gracefully vacate them whenever the returning incumbent (or primary) user wants to use his/her band. This spawns another question of what type of radios are these networks made up of? And the answer is, radios that are capable of cognition, radios that are aware of their environment, Cognitive Radios \cite{Haykin05cognitiveradio} has been a buzz word ever since radios have evolved to what they call ``the software defined radio''. Thus, by exploiting cognitive radio networks with dynamic spectrum access techniques efficient spectrum utilization is a possibility. But, before reaching that far, one has to understand the primary user's spectrum occupancy patterns over a period of time at a particular place. In this report, we try to understand the patterns of spectrum occupancy in Blacksburg, VA in selected bands which consistently vary according to a particular event. We also take this opportunity to compare the occupancy in both Blacksburg and Chicago to find some spectrum bands showing interesting occupancy variations.

	\subsection{Motivation and Goal}
	
	The potentiality of finding vacant spectrum in a decently sized university town like Blacksburg (population over 50,000 \cite{bbDemo}) makes it the best candidate to experiment Dynamic Spectrum Access mechanisms over these seemingly empty bands. This leads to a curiosity to look through the eyes of a RF spectrum monitoring device, filtering out those bands that vary in their occupancy over a period of time. This has been the primary motivation towards carrying out such a broad based spectrum study. The objective of this broad based spectrum occupancy measurement study is gaining insightful information on current trends of primary user occupancy in selected bands over arbitrary weekday(s), regular weekend(s) and game day weekends and then drawing conclusions based on the measurements and occupancy plots. These studies can provide an excellent baseline to mathematically model primary users' occupancy in selected bands.
	
	\subsection{Procedure}
	
	In order to meet with our objective, we first begin with the installation of our RF Spectrum Observatory here at Virginia Tech. We briefly discuss the installation and configuration of our spectrum observatory in Section 3. Once our observatory is up and running, we collect required measurements for arbitrarily selected regular weekdays, regular weekend and on game day weekends. In the section 4, we use these measurement data to plot three essential plots,
	\begin{enumerate}
		\item[(a)] Power Spectrogram
		\item[(b)] Min-max-mean Power Spectral Density with threshold
		\item[(c)] Occupancy Charts
	\end{enumerate}
	We also combine occupancy charts for the three unique days under consideration. By doing a band-by-band comparison we extract those bands where the variations in user occupancy is consistently distinct for a particular event (like Virginia Tech game day). In section 5, we draw conclusions on the basis of the variations observed.

\pagebreak

\section{Related Work}

The very first large scale spectrum measurement campaign was conducted in the year 1998 \cite{750342} by the National Telecommunications and Information Administration (NTIA), a Federal agency that is responsible to regulate the usage of government held spectrum in the United States. The work done by \textit{Wellens et. el} \cite{4549835}, talks about the spectrum occupancy measurement campaign conducted in the city of Aachen, Germany. They have briefly described their experience with the installation of spectrum monitoring system capable of carrying out spectrum measurements from 20 MHz to 6 GHz. They have used Amplitude Probability Distribution (APD), to investigate primary user activity. Another major measurement campaign led by \textit{McHenry et. el} \cite{McHenry:2006:CSO:1234388.1234389} through Shared Spectrum Company involves carrying out spectrum occupancy measurements at different locations \cite{SSCSpecReports}. In these reports, the SSC researchers have carried out occupancy measurements from 30 MHz to 2900 MHz and have plotted maximum power spectral density, power spectrogram, waterfall plots and duty-cycle occupancy charts.  In some of these reports they have compared the occupancy numbers per-band between two different places. From these reports, we can see that average occupancy in a rural setup is typically between 1\% and 3.4\% and that of an urban setup is typically between 11.4\% and 17.4\%. A very critical aspects of these measurements is the size of data-set used for comparison fairly varies over different places and as it is not coherent. This becomes questionable whether the per-band occupancy between two places comparable. Moreover, surrounding events on the day of the measurement also matters the most as the occupancy at any place is conditioned to these surrounding events at that time. Our work in this report follows the same procedure except that we primarily focus on those bands which show consistent variation pattern on Virginia Tech game days over regular college days. 

Virginia Tech spectrum observatory installation and the spectrum occupancy assessment study forms a part of a Global RF Spectrum Opportunistic Assessment project, a WiFiUS initiative \cite{HAGER_GRANT}. Our partners in this project, Illinois Institute of Technology, Chicago and VTT, Turku, Finland have successfully installed the spectrum observatory and have been carrying out measurements from almost a year. We sought help from our partners during the installation and configuration of our spectrum observatory. The reference \cite{6849666}, talks about the spectrum observatory installation and comparison of occupancy measurements carried out over these places.

% (involving Chicago Spectrum Occupancy Measurements \& Analysis and a long term Studies Proposal, Mark McHenry's paper)

% Evaluation of Spectrum Occupancy in Spain for Cognitive Radio Applications

\pagebreak

\section{Virginia Tech Spectrum Observatory System}

\subsection{RF Spectrum Monitoring System Installation}

\begin{figure}[h!]
  \centering
    \includegraphics[width=0.85\textwidth]{observatory_high_level_design.png}
	\caption{High Level System Design}
\end{figure}

The Virginia Tech Spectrum Observatory was installed during May 2014 and it has been continuously monitoring RF spectrum from 30 MHz to 6 GHz since then. Figure 1 shows a high level system design of our spectrum observatory. The spectrum observatory has the following essential components:
\begin{enumerate}
	\item[a.] \textbf{CRFS' RFEye Spectrum Sensor Node} - This is a core component and is capable of acquiring real-time spectrum data. The RF sensor node has the following hardware specifications:
	\begin{enumerate}
		\item[i.] The in-built receiver has a range of 10 MHz to 6 GHz (can be extended up to 18 GHz with a Block down converter).
		\item[ii.] It has a noise figure of typically 8 dB and 11 dB for 10 MHz - 4 GHz and 4 GHz - 6 GHz respectively.
		\item[iii.] The receiver's spurious free dynamic range is at least 60 dB. More information on the RFEye receiver specification can be found at \cite{rfeye_specs}.
	\end{enumerate}
	Among the interfaces, we have, 
	\begin{enumerate}
		\item[i.] Four RF SMA inputs that aid in capturing real-time spectrum data, 
		\item[ii.] Two USB  2.0 ports (currently unused) that can be used to connect to local storage devices like flash drives for back-up and collecting logs, 
		\item[iii.] GPS and cellular modem SMA interfaces, out of which the GPS port is currently connected to synchronize the system time with the UTC time. 
		\item[iv.] Finally, the most important interface which is the 1 Gbps Power On Ethernet (PoE) port through which the node is currently powered up. It is the only gateway through which the node can be accessed and configured. The Ethernet port is also used to acquire spectrum data and store it to an external network hard drive. Hence, it forms the only point of contact to the RFEye node.
	\end{enumerate}
	
	From a software perspective, the RFEye node runs a Linux based operating system with full C and Python development environment available. The system runs an in-built proprietary web-server designed by the company. The web interface for this server can be accessed through a regular browser. There are built-in proprietary applications that can be activated/deactivated, helping us in node configuration and data collection.
	Some of the essential applications that are currently used to configure and collect data are:
	\begin{enumerate}
		\item[i.] NCPD, Control protocol server - This application helps in initially configuring the node, where we set the node information, RF input information and clock settings. We parse these information in a dictionary format while modifying the key values. One may find the current NCPD configuration in the appendix section.
		\item[ii.] GPSD, GPS application - This application helps in configuring the parameters of in-built GPS receiver, like the debug level, position state (fixed or moving), communication port, baudrate, etc.
		\item[iii.] LOGGER, logger application - This application forms the most important when it comes to storing of spectrum data. It also provides a highly flexible configuration file through which a well structured band plan can be implemented. More on logger configuration will be discussed in the subsection, ``Configuration of RFEye node''.
	\end{enumerate}
	
	\item[b.] \textbf{MP Antenna Super-M Ultra Base} - This is an ultra wide band spider antenna with a receive frequency range from 25 MHz to 6 GHz. The antenna is clamped to a seven feet high mast and has a N-type jack and connects to the input of the three way splitter.
	\item[c.] \textbf{Three way splitter} - The signal received by the antenna goes to the input of the three way splitter. Out of the three splitted outputs, two of them are filtered and fed into the input of the RFEye node. The third one is directly fed into the input of the RFEye node. 
	\item[d.] \textbf{300 MHz high pass filter} - One of the splitted output from the three-way splitter is high pass filtered at 300 MHz to remove interference due to VHF signals.
	\item[e.] \textbf{3 GHz high pass filter \& high gain wideband amplifier} - One of the splitted output from the three-way splitter is processed through a 3 GHz high pass filter which is in turn fed into a high gain wide band amplifier. The output signal from the amplifier is fed into as one of the input of the RFEye node.
	\item[f.] \textbf{Network hard drive} - We have used a 3 TB Western Digital network hard drive for remotely storing the spectrum data acquired by the RFEye node.
\end{enumerate}

\subsection{Configuration of RFEye node}
	
	The logger application accepts a configuration file which provides complete information on spectrum data storage path and scans configured according to the band plan. We can configure the location of the log file, the location of the data directory (where the spectrum data will be stored), file size, log level, etc. We can also configure each scan by setting the parameters like start frequency, stop frequency, sweep time and resolution bandwidth. Once the logger configuration file is prepared, it is fed through the web interface and the logger application is activated. With the help of web interface, we can get to know the status of the logger application.

\subsection{Selecting an appropriate location}
	
	The observatory was installed on the roof-top of Whittemore hall. This location plays a very important role as it is less than a mile from the Downtown Blacksburg and almost a mile far from Lane stadium, the place where Virginia Tech college football is being held. Figure 2 shows the observatory location in Google maps with respect to the downtown Virginia Tech and lane stadium.
	
	\begin{figure}[ht!]
  \centering
    \includegraphics[width=0.95\textwidth]{observatory_site.png}
		\caption{Observatory installed at Whittemore Hall's roof-top}
	\end{figure}

\pagebreak

	% Explanation about the band-plan
%%% ******************************** Complete till here ************************************************* %%%

\section{Spectrum Measurements and Occupancy comparison}

In this section we will use the collected measurement data from the spectrum observatory and carry out an occupancy comparison. Our first subsection will deal in finding out those bands which vary in their occupancy over Virginia Tech football game with respect to a regular weekday and a regular weekend. We also swept through different bands from 30 MHz to 3 GHz that possibly change due to the football event.

In the second subsection we repeat the same exercise of occupancy comparison with Blacksburg and Chicago's spectrum data. The measurements in Chicago were collected using a similar spectrum observatory situated on the rooftop of the IIT tower.
We arbitrarily selected the same weekday, Aug 27, 2014 for Blacksburg and for Chicago. We then scanned through the spectrum 30 MHz - 3 GHz spectrum to find out those bands which show a significant occupancy variation.

The thresholds for the occupancy measurements were carefully selected by plotting the minimum, maximum and mean power spectral density. For the LMR band comparison, we chose a threshold of -93 dBm for the Blacksburg data and -109 dBm for the Chicago measurements.

	\subsection{Broad based RF measurements and occupancy comparison for a regular weekday, regular weekend day and a weekend Virginia Tech game day}
	
	In this subsection we will be comparing the spectrum bands which show significant variation in their occupancy during Virginia Tech football game weekends with respect to a regular college day and a regular weekends. We arbitrarily chose the fall 2014 season's Virginia Tech first college football which was held on Aug 30, 2014. We also picked arbitrary weekday Aug 27, 2014 and a regular weekend on Oct 11, 2014. As we scan through the RF spectrum, we notice some variations in the following band:
	
		\subsubsection{Measurement comparison for the LMR band of 450 MHz to 470 MHz}
		
		From figure 3, we can see that Virginia Tech game day has a higher occupancy than the regular weekday. Also, during regular weekend, the occupancy is much lower than a regular weekday or a game day weekend due to minimal usage of public safety systems. We use markers to point the frequency segments that distinctly show a higher occupancy on a game day. We further investigate the reasoning behind the surge in the occupancy using Blacksburg's LMR spectrum database \cite{blacksburg_spec_occ} which gives detailed information about the incumbent users and services provided by these primary users per channel.
		
		\begin{figure}[ht!]
			\centering
				\includegraphics[width=0.95\textwidth]{LMR_comparison_gameday.png}
			\caption{450 MHz - 470 MHz LMR band VT game day occupancy comparison}
		\end{figure}
		
		\begin{enumerate}
			\item[(a)] Marker 1 shows a higher occupancy in channel frequencies that provide school related services on the game day. For e.g., 451.8625 MHz provides parking services and 452.2 MHz is used by Virginia Tech for special events. It is quite obvious that parking on a game day is more busier than a regular day which in turn is busier than a regular weekend. There is also a surge in the occupancy of Montgomery County's sheriff's dispatch which operates at 452.3 and 452.35 MHz.
			\item[(b)] Marker 2 also shows a higher occupancy in channel frequencies that provide school related services on the game day. For e.g., 461.7875 and 461.8875 MHz provides Virginia Tech Emergency Medical Services (EMS) and facility services. The Virginia Tech athletics service is provided at frequencies 463.4875 and 464.5875 MHz which obviously shows a surge on the Virginia Tech game day with respect to regular weekday and a regular weekend.
		\end{enumerate}
		
		Thus we see a perceivable change in occupancy in those LMR channels which are affiliated with the event and also the services that are influenced either directly or indirectly by the event.
		
		Apart from browsing other bands for variations in game day occupancy, we expected that the cellular band would be a potential candidate for occupancy comparison. However after going through the measurement plots of cellular band's uplink, the occupancy numbers were misleading because these handheld devices transmit at a very low power and by using a simple energy detector to get the occupancy numbers, we would certainly miss a lot of uplink transmissions. Also for our cellular downlink measurements, we didn't find any variations in the power levels on a game day weekend or a regular weekday or a regular weekend. 
 	
	\subsection{Broad based RF measurements and occupancy comparison for a regular weekday in Blacksburg, VA and Chicago, IL}
	
	In this subsection we will be peeking into the spectrum bands of both Chicago and Blacksburg which show significant differences in terms of their occupancy. We also try to add up as much information about the incumbent users operating in these bands, the wireless technology they are using over these bands.
	
		\subsubsection{Scanning through the Land Mobile Radio (LMR) band}
		
		\begin{figure}[ht!]
			\centering
				\includegraphics[width=0.95\textwidth]{LMR_comparison_Blacksburg_Chicago.png}
			\caption{Duty cycle plot from 450 MHz to 470 MHz LMR band for Blacksburg and Chicago occupancy comparison}
		\end{figure}
		
		Our first candidate LMR band which grabbed our attention is, the LMR band ranging from 450 MHz to 470 MHz. Figure 4 shows the duty cycle plot comparing spectrum occupancy in Blacksburg and Chicago for the weekday, August 27, 2014. From the figure, the first thing that comes into our mind is that, the bands in Chicago show a heavier occupancy over Blacksburg which is quite obvious. We will take the help of frequency database found in \cite{chicago_spec_occ} \cite{blacksburg_spec_occ}, where we find information on channel allocations, services offered, technology used, etc. There are numerous services that are being offered in this LMR band especially in the city of Chicago. Thus, we have put markers comparing both Blacksburg and Chicago to explain those frequency segments that interests us.
		
		\begin{enumerate}
		
			\item[(a)] Marker 1 shows around 10\% occupancy in Blacksburg and around 2\% occupancy in Chicago for the channel occupying the frequency segments 450.8 - 451.4 MHz. In Blacksburg, the channels in 450.8 - 451.4 MHz segment having center frequencies 451.1, 451.275 and 451.375 MHz are used by the Town of Blacksburg's fire dispatch, tactical command (tac) and Montgomery County's sheriff respectively. In Chicago, school services and public works telemetry services are offered at the frequencies 451.1875 (4), 451.1125, 451.1375 (2), 451.1625 and 460.175 (2). The round brackets with a number in it indicates that these frequencies are being spatially reused.
			Though there are services offered in Chicago, the figure 2 notably shows lesser occupancy as compared to Blacksburg in the 450.8 - 451.4 MHz segment.
			
			\item[(b)] Marker 2 shows comparatively higher occupancy for Chicago than Blacksburg. We compare the channels allocated and services provided within each frequency segment which are pointed by these markers:
			
			\begin{enumerate}
				
				\item[(i)] In Chicago the channels in the 451.75 - 452.5 MHz segment are used mainly for Trunked Radio System (serves multiple services with the same channel allocation), school maintenance. operations and services. For e.g., the channel frequencies 451.7675 MHz is used for operations in schools located in Riverside and Brookfield, 461.81250 MHz for security, operations and maintenance of Proviso Township High School. The segment also provides channels that are allocated for private businesses. For e.g. the channel at 451.85 MHz is also used in Cargo/Shipping companies and the channel at 452.125 MHz is used for Northwest airlines ground operations. This segment also provides transportation services. For e.g. channel frequencies 452.3625, 452.3875 MHz  are used by Taxi-cab.
				
				In Blacksburg, the channels in the 451.75 - 452.5 MHz segment are used by Virginia Tech Parking services at 451.8625 MHz, Montgomery County's Fire tac which operates at 451.7875 MHz and Virginia Tech's Communication Network Services operating at 452 MHz. This segment also offers EMS services at 452.425 and 452.25 MHz, provided by the Riner Rescue Squad and Christiansburg Rescue respectively. Apart from this, the school provides services during special event (like Virginia Tech football game) at 452.2 MHz.
				
				\item[(ii)] In Chicago, the channels in the 453 - 453.5 MHz segment are used for services like transportation and public works. For e.g., Chicago Transit Authority uses channel frequencies, 453.225, 453.375, 453.425 and 453.475 MHz to provide wireless communications for their running buses. The parking garage services at Evanston uses 453.0625 and 453.0875 MHz. The Metropolitan water reclamation district uses channels having center frequencies 453.275 and 453.3375 MHz.
				
				In Blacksburg, the channels in the 453 - 453.5 MHz segment are used for school services, county's and town's law and order. For e.g. Blacksburg Police Department's dispatch services are handled at 453.4875 MHz.
				
			\end{enumerate}
			
			\item[(c)] Marker 3 shows relatively higher occupancy for Blacksburg than Chicago in channels that are in the frequency range of 453.6 - 454 MHz. In Blacksburg, Virginia Tech Police Department operates dispatch services at 453.875 MHz, Blacksburg Transit Authority uses 453.625 and 453.85 MHz for their operations. The Montgomery County's sheriff also offers law and order services in this frequency segment. In Chicago, there are few allocations in this segment and they provide services mostly related to public works. For e.g. the ground maintenance services offered in North Berwyn Park District at 453.6375 MHz.
			
			%\item[(d)] Marker 4 shows heavy occupancy only in Chicago area and almost no occupancy in Blacksburg. The channel segments that fall in this category are 454 - 456 MHz, 462.7 - 463.4 MHz.
			%
			%\begin{enumerate}
				%\item[(i)] The channel segment
				%\item[(ii)] The channel segment
			%\end{enumerate}
			
			\item[(d)] From marker 4 we can see that there is a higher occupancy in Blacksburg centered at 460 MHz. This is because of an unknown incumbent user centered at 460 MHz. In Chicago, its Police Department uses the band ranging from 460 to 460.6 for their operations citywide. The EMS teams also use this band for their operations in Chicago city. Thus, we can see relatively a higher occupancy from 460 to 460.6 MHz band in Chicago than in Blacksburg.
			
	\end{enumerate}
	
	Thus, looking at the overall occupancy numbers which is 4.06\% and 10.09\% in Blacksburg and Chicago respectively, it is quite obvious as we can see that the city of Chicago has a higher channel allocations \cite{chicago_spec_occ} which in turn has a heavier occupancy as compared to Blacksburg. Also, some channel segments in this 450 - 470 MHz have an active Trunked Radio Services which further increases the occupancy of the channels in those frequency segments as compared to Blacksburg. But, from markers 1 and 3, we can see that there are some segments where Blacksburg's channel occupancy outnumbers Chicago's. This is because, in Blacksburg, these marked channel segments provide crucial services related to the law and order, transportation, etc. unlike public works related services provided in Chicago which leads to low occupancy.
	
	Our second candidate spectrum band that has a higher spectrum allocations and occupancy in Chicago is the 700 MHz LMR band. The 700 MHz Public Safety band extends from 763 - 775 MHz and 793 - 805 MHz \cite{fcc_700_MHz_lmr}. This band is divided into three segments \cite{fcc_700_MHz_lmr}:
	
	\begin{enumerate}
		\item[(i)] 763 - 768/793 - 798 MHz segment allocated for public safety broadband communications
		\item[(ii)] 768 – 769/798 – 799 MHz segment is reserved as a guard band
		\item[(iii)] 769 – 775/799 – 805 MHz segment allocated for public safety narrowband communications - The two narrowband segments, 769 - 775 MHz are used for base operations and 799 - 805 MHz used for mobile operations. Each narrowband segment is divided into 960 channels, with each channel having a size of 6.25 kHz.
	\end{enumerate}
	
	\begin{table}[ht!]
	\centering
		\begin{tabular}{| p{1cm} | p{6cm} | p{4cm} |}
		\hline
		Sr. No. & Service Name & Frequencies used (in MHz)	\\	\hline
		1. & Cook County's Sheriff's Police Department & 769.5125, 773.1375, 774.9125, 769.2625, 769.8125, 770.2625, 771.2625, 771.5375, 771.9875, 772.3875, 773.3875, 773.6375 \\ \hline
		2. & Pace Suburban Bus Service of the RTA & 772.64375	\\	\hline
		3. & DuPage County's Emergency Telephone System Board (ETSB) & 769.33125, 769.75625, 770.05625, 770.80625	\\
		\hline
		\end{tabular}
		\caption{769 - 775 MHz LMR allocations in Chicago city}
	\end{table}
	
	\begin{figure}[ht!]
		\centering
			\includegraphics[width=0.95\textwidth]{700_MHz_LMR_comparison_Blacksburg_Chicago.png}
		\caption{Duty cycle plot from 769 MHz to 774 MHz LMR band for Blacksburg and Chicago occupancy comparison}
	\end{figure}
	
	But, while going through these public safety bands, the narrowband segment 769 – 775 MHz appeared to be quite interesting. As we can see in figure 5, this narrowband segment shows a heavy occupancy in Chicago city and almost no occupancy in the Blacksburg area. The primary incumbents who operate in these bands are, Cook County’s sheriff’s Police Department, Pace suburban bus service of the RTA and DuPage County’s Emergency Telephone System Board. The table 1 shows the frequency operating at the particular frequencies.
	
	The markers in the figure 5 correspond to the serial number in table 1. Thus we can see a higher occupancy of the 700 MHz LMR spectrum in urban areas which is not at all used in a college town like Blacksburg.
	
	\subsubsection{Comparison of occupancy in Television bands}
	
	\begin{table}[h!]
		\begin{tabular}{| p{2cm} | p{2cm} | p{3cm} | p{2cm} | p{2cm} | p{2cm} | p{2cm} |}
		\hline
		RF channel \#	& Virtual channel \#	& Frequency Range (In MHz)	& Callsign 	& Primary User 	& Location (in VA) & Power (in kW) \\	\hline
		
		3 &	15 & 60 - 66 & WBRA-TV & PBS & Roanoke & 8 \\ \hline
		% Occupied
		13 & 13 & 234 - 240 & WSET-TV &	ABC & Lynchburg	& 28.7 \\ \hline
		% Not Occupied
		17 & 27 & 488 - 494 & WFXR & Fox & Roanoke & 695 \\ \hline
		% Occupied
		18 & 7 & 494 - 500 & WDBJ & CBS &	Roanoke &	460 \\	\hline
		% Occupied
		20 & 21 & 506 - 512 & WWCW & CW & Lynchburg	&	916 \\	\hline
		% Not Occupied
		30 & 10 & 566 - 572 &	WSLS-TV &	NBC &	Roanoke & 1000 \\	\hline
		% Occupied
		36 & 38 & 602 - 608 & WPXR-TV & ION & Roanoke & 700 \\
		% Occupied
		\hline
		\end{tabular}
		\caption{TV channel information near Blacksburg, VA}
	\end{table}
	
	We next compare the spectrum allocation and occupancy of Television (TV) band, both in Chicago and in Blacksburg. The reception of TV signals in the town of Blacksburg is from the nearby city, Roanoke which is 40 miles far. Whereas Chicago city has higher number of TV stations within the city limits and nearby. Table 2 enlists the TV channels that are possibly received in Blacksburg \cite{tv_channel_info_blacksburg} \cite{tv_channel_frequencies}. Similarly, table 3 lists the TV channels that are possibly received in Chicago \cite{tv_channel_info_chicago} \cite{tv_channel_frequencies}. These tables will just give us the information about the primary incumbents of the particular channel. But with the help of the power spectrogram plots, we compare occupancy of TV bands both in Blacksburg and Chicago area.
	
	\begin{table}[ht!]
		\begin{tabular}{| p{2cm} | p{2cm} | p{3cm} | p{2cm} | p{2cm} | p{2cm} | p{2cm} |}
		\hline
		RF channel \#	& Virtual channel \#	& Frequency Range (in MHz)	& Callsign 	& Primary User 	& Location 	& Power (in kW) \\	\hline
		
		12 & 2 & 204 - 210 & WBBM-TV & CBS & Chicago, IL & 8 \\	\hline
		% Occupied
		29 & 5 & 560 - 566 & WMAQ-TV & NBC & Chicago, IL & 350 \\	\hline
		% Occupied
		44 & 7 & 650 - 656 & WLS-TV & ABC & Chicago, IL & 1000	\\	\hline
		% Occupied
		19 & 9 & 500 - 506 & WGN-TV & CW & Chicago, IL & 645 \\	\hline
		% Occupied
		47 & 11 & 668 - 674 & WTTW & PBS & Chicago, IL & 300 \\	\hline
		% Occupied
		21 & 20 & 512 - 518 & WYCC & PBS & Chicago, IL & 98.9	\\	\hline
		% Occupied
		27 & 26 & 548 - 554 & WCIU-TV & Independent station & Chicago, IL & 160 \\	\hline
		% Occupied
		31 & 32 & 572 - 578 & WFLD & Fox & Chicago, IL & 1000	\\	\hline
		% Occupied
		43 & 38 & 644 - 650 & WCPX-TV & ION & Chicago, IL & 200 \\	\hline
		% Occupied
		45 & 44 & 656 - 662 & WSNS-TV & TEL & Chicago, IL & 467 \\	\hline
		% Occupied
		51 & 50 & 692 - 698 & WPWR-TV & MNT & Gary, IN & 1000	\\	\hline
		% Occupied
		50 & 60 & 686 - 692 & WXFT-DT & TF & Aurora, IL & 172	\\	\hline
		% Occupied
		36 & 62 & 602 - 608 & WJYS & Independent station & Hammond, IN & 50	\\	\hline
		% Occupied
		38 & 66 & 614 - 620 & WGBO-DT & UNI & Joliet, IL & 600	\\	\hline
		% Occupied
		17 & 56 & 488 - 494 & WYIN & Northwest Indiana Public Broadcast & Gary, IN & 300 \\
		% Occupied
		\hline
		\end{tabular}
		\caption{TV channel information near Chicago, IL}
	\end{table}
	
	\begin{figure}[ht!]
		\centering
			\includegraphics[width=0.95\textwidth]{470_512_TV_Band_Comparison.png}
		\caption{Power Spectrogram from 470 to 512 MHz for the entire day of Aug 27, 2014 comparing Chicago and Blacksburg}
	\end{figure}
	
	As we can see that we have shown the power spectrogram plots comparing Blacksburg and Chicago from 470 to 698 MHz in figure 6. From the figure 6, we can see that for Blacksburg, the channel number 17 is occupied. From table 2 we can check that the RF channel 17 is operated by Fox in Blacksburg. The TV transmitter for this station is located in Roanoke. Also, from the figure we can see that the same channel is being operated by Northwest Indiana Public Broadcast company and the transmitter is located in Gary, IN which is 40 miles away from the city of Chicago. 
	Similarly, for Blacksburg, we can see from the figure that the RF channel 18 is also occupied by CBS. In Chicago, the same TV channel is unoccupied. We can also see that the RF channel 20 is unoccupied in Blacksburg though it operated by CW Television Network. Another channel from Lynchburg i.e. RF channel 13 operated by ABC is also not received in Blacksburg (not shown in figure). Both of these TV channels' transmitters are located in Lynchburg which is roughly 60 miles far from Blacksburg. 
	
	\begin{figure}[ht!]
		\centering
			\includegraphics[width=0.95\textwidth]{512_608_TV_Band_Comparison.png}
		\caption{Power Spectrogram from 512 to 608 MHz for the entire day of Aug 27, 2014 comparing Chicago and Blacksburg}
	\end{figure}
	
	\begin{figure}[ht!]
		\centering
			\includegraphics[width=0.95\textwidth]{608_698_TV_Band_Comparison.png}
		\caption{Power Spectrogram from 608 to 698 MHz for the entire day of Aug 27, 2014 comparing Chicago and Blacksburg}
	\end{figure}
	
	We have also shown the power spectrogram plots comparing Blacksburg and Chicago from 512 to 608 MHz and 608 to 698 MHz in figure 7 and 8 respectively. While comparing both these figures, we find that almost all the channels listed in the table 2 and table 3 are found occupied. The location of the transmitters for TV channels WXFT-DT, WJYS and WGBO-DT are all within the 50 miles radius from Chicago city.
	
	Also, from figure 7, RF channels 25, 26, 30 and 32 are seemingly occupied in Chicago but the primary users information is not known. Also, from figure 8 channels 39 and 49 are found occupied but their PUs information is not known.
	
	Thus it is pretty obvious from the above observation that the number of channels occupied in Chicago is definitely higher than occupied in Blacksburg, which means more whitespace availability in Blacksburg as compared to Chicago.
\pagebreak
	
	%NOTE: Distance of these cities from Chicago city:
	%
	%Gary, IN (40 miles from Chicago city)
	%Aurora, IL (41.5 miles from Chicago city)
	%Hammond, IN (25 miles from Chicago city)
	%Joliet, IL (40 miles from Chicago city)
	%LaSalle, IL (100 miles from Chicago city)
	
	%\subsubsection{Peeking into the highly congested cellular band}
	%
	%The cellular band includes that were studied in this measurement comparison exercise are:
	%\begin{enumerate}
		%\item[(i)] 700 MHz cellular band: The 700 MHz cellular band includes 700 - 750 MHz, 750 - 763 MHz, 775 - 793 MHz
		%\item[(ii)] PCS band: The PCS band includes bands from 1850 to 1910 MHz and from 1930 to 1990 MHz.
		%\item[(iii)] AWS band: The AWS band includes band
	%\end{enumerate}
	%We scanned through these bands by following the FCC spectrum dashboard \cite{fcc_dashboard} which provided a decent idea about the primary users. But, we couldn't get much information on what technology they are operating on what band. From \cite{lte_spectrum}, which says 
	
	%\subsubsection{Other bands}
	%
	%One such band that caught my interest while sweeping through the RF spectrum from 30 MHz to 3 GHz is the Paging and Radio location band. 
	

\section{Conclusion}

In a nutshell, we found significant variation in occupancy for the 450 - 470 MHz LMR band during Virginia Tech football game weekend with respect to regular weekends and regular weekdays. From this comparison, we conclude that, the occupancy variations are distinctly higher in the channels that provide school related services which are directly influenced by the game event.

Also, due to overwhelming number of services provided in Chicago's 450 - 470 MHz LMR band, particularly trunked radio services in action, we see a heavy occupancy in most of the LMR channels as compared to Blacksburg. From the 700 MHz LMR band, we conclude that the band shows a higher occupancy in urban areas as compared to rural areas where the band is almost empty. Lastly, we also saw a higher occupancy in the TV bands in Chicago than in Blacksburg, thereby concluding less availability of whitespace in urban areas as compared to rural areas which is quite obvious.

\pagebreak
\section*{Appendix}

\subsection*{Current NCPD configuration}

\begin{lstlisting}

{
	"version":1,
	"information":
		{
			"node_name": "RFEYE00XXXX",
			"node_description": "Blacksburg Virginia Tech WiFi-US Spectral Analysis",
			"update_user": "WiFi-US Virginia Tech",
			"update_time": "06:35"
		},
	"settings":
		{
			"real_time_clock":"gps",
			"reference_clock":"gps",
			"gps_time_offset": -5
		},
	"antennas":
		[
			{
				"uid": 2,
				"name": "Input 2",
				"description": "SUPER M ULTRA BASE ANTENNA #1 - 3GHz HPF and amplified signal",
				"rf": 2,
				"range": {"min": 10, "max":6000}
			},
			{
				"uid": 3,
				"name": "Input 3",
				"description": "SUPER M ULTRA BASE ANTENNA #1 - raw signal from splitter",
				"rf": 3,
				"range": {"min": 10, "max":6000}
			},
			{
				"uid": 4,
				"name": "Input 4",
				"description": "SUPER M ULTRA BASE ANTENNA #1 - 300 MHz HPF signal",
				"rf": 4,
				"range": {"min": 10, "max":6000}
			}
		]
}

\end{lstlisting}


%\subsection{Power spectrogram of TV Bands ranging 57 - 406 MHz comparing Chicago and Blacksburg}
%
%
%
%\subsection{Power spectrogram of Cellular Bands comparing Chicago and Blacksburg}


%%%%%%%%%%%%%%%%%
%
% Include an EPS figure with this command:
%   \epsffile{filename.eps}
%

%%%%%%%%%%%%%%%%
%
% Do tables like this:

 %\begin{table}
 %\caption{The Graduate School wants captions above the tables.}
%\begin{center}
 %\begin{tabular}{ccc}
 %x & 1 & 2 \\ \hline
 %1 & 1 & 2 \\
 %2 & 2 & 4 \\ \hline
 %\end{tabular}
%\end{center}
 %\end{table}

%%%%%%%%%%%%%%%%%%%%%%%%%%%%%%%%

% If you are using BibTeX, uncomment the following:
% \thebibliography
%
% Otherwise, uncomment the following:
% \chapter*{Bibliography}

% \appendix

% In LaTeX, each appendix is a "chapter"
% \chapter{Program Source}

\bibliography{reference}

\end{document}