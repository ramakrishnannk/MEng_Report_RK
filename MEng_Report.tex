%
% PROJECT: A Broadbased RF spectrum occupancy comparison of Blacksburg and Chicago
%   TITLE: LaTeX report template for ETDs in LaTeX
%  AUTHOR: Ramakrishnan Kalyanaraman, rk126@vt.edu
%     URL: http://etd.vt.edu/latex/
% SAVE AS: MEng_Report.tex
% REVISED: December 18, 2014 [GMc 8/30/10]
% 

% Instructions: Remove the data from this document and replace it with your own,
% keeping the style and formatting information intact.  More instructions
% appear on the Web site listed above.

\documentclass[12pt,sts]{report}

\setlength{\textwidth}{6.5in}
\setlength{\textheight}{8.5in}
\setlength{\evensidemargin}{0in}
\setlength{\oddsidemargin}{0in}
\setlength{\topmargin}{0in}

\setlength{\parindent}{0pt}
\setlength{\parskip}{0.1in}

\usepackage{enumerate}

% Uncomment for double-spaced document.
% \renewcommand{\baselinestretch}{2}

\usepackage{setspace}
\usepackage{url}
\usepackage{float}

\usepackage[english]{babel}
\usepackage{graphicx}
\usepackage{listings}
% \floatstyle{boxed}
% \restylefloat{figure}

\doublespacing
% \usepackage{epsf}
\setcounter{tocdepth}{5}
\begin{document}

\thispagestyle{empty}
\pagenumbering{roman}
\begin{center}

% TITLE
{\Large 
RF Spectrum Occupancy Survey of Blacksburg, VA
}
% Capital S in Survey

\vfill

\textbf{Ramakrishnan Kalyanaraman} \\
Master of Engineering \\
in \\
Computer Engineering \\
Bradley Department of Electrical and Computer Engineering \\
Virginia Polytechnic Institute and State University \\
Email: rk126@vt.edu

\vfill

\textbf{Committee Members} \\
% Capital M in Members
Dr. Allen B. MacKenzie (Co-chair) \\
Dr. Luiz A. DaSilva (Co-chair) \\
Dr. Jeffrey H. Reed
% Indicated both Dr. MacKenzie and Dr. DaSilva as Co-chair

\vfill

January 26, 2014 \\
% Changed date to defense date
Blacksburg, Virginia

\end{center}

\pagebreak

\thispagestyle{empty}

\begin{center}

\textbf{ABSTRACT}

\end{center}

In this report we discuss a broad-based RF spectrum survey conducted at Virginia Tech in Blacksburg, VA. The measurements for this spectrum study were collected from Virginia Tech's Spectrum Observatory, and the installation of this observatory is also discussed in this report. We compare the spectrum occupancy for arbitrary weekdays, regular weekend days and college football weekend days, looking for consistent occupancy patterns. We report primarily on those bands which show interesting, explainable variations in spectrum occupancy over these periods. We also compare the spectrum usage in Blacksburg, VA to the spectrum usage in Chicago, IL on an arbitrarily chosen weekday.

\vfill

\pagebreak

% Dedication and Acknowledgments are both optional
% \chapter*{Dedication}
% \chapter*{Acknowledgments}

\tableofcontents

\pagebreak

\listoffigures
\pagebreak

\listoftables
\pagebreak

\section*{Acknowledgments}

I am immensely grateful to my committee's co-chairs, Dr. Allen B. MacKenzie and Dr. Luiz A. DaSilva, and also my committee member, Dr. Jeffrey H. Reed, for providing excellent support and guidance, thereby encouraging me throughout my graduate school.

I would like to extend my acknowledgment to our project partners in Finland and Chicago for providing information on setting up the Virginia Tech Spectrum Observatory. I would like to thank my lab members Mr. Abdallah S. Abdallah and Mr. Cameron W. Patterson for providing help in the installation of the observatory.

Finally, I thank my family and friends for their extended support.
\pagebreak

\pagenumbering{arabic}
\pagestyle{myheadings}

\renewcommand\thesection{\arabic{section}}

\section{Introduction}
% \setcounter{section}{1}

For the past two decades, there has been an increase in the number of spectrum measurement campaigns that are being carried out in an effort to survey spectrum occupancy in both urban and rural areas. Researchers have also sought to develop mathematical models characterizing spectral usage. Most of these studies have shown that numerous spectrum bands lie vacant or underutilized, though licensed by the regulatory bodies. Recently, Dynamic Spectrum Access (DSA) has emerged as one plausible technique to improve spectrum utilization \cite{IEEEexample:Akyildiz06nextgeneration}. What is Dynamic Spectrum Access? DSA enables opportunistic access to sparsely occupied bands for secondary users. These secondary users are licensed to use these bands, but they must vacate them whenever the incumbent (or primary) user wants to use the band. The operation of a DSA system will be aided, though, by an understanding of the primary user's spectrum occupancy patterns over time at a particular place. In this report, we try to understand the patterns of spectrum occupancy in Blacksburg, VA in selected bands which vary according to a particular event. We also compare the occupancy in Blacksburg with the occupancy in Chicago and find some spectrum bands showing interesting occupancy variations.

	The objective of this broad-based spectrum occupancy measurement study is to gain insight into the occupancy of selected bands over weekdays, regular weekend days, and game days and then drawing conclusions based on the measurements and occupancy plots.
	% These studies can provide an excellent baseline to mathematically model primary users' occupancy in selected bands.
	
	In order to meet our objective, we first installed an RF Spectrum Observatory at Virginia Tech. We briefly discuss the installation and configuration of the spectrum observatory in Section 2. Once the observatory was running, we collected required measurements for arbitrarily selected weekdays, regular weekend days, and game days. In the section 3, we use these measurements to create three essential plots:
	\begin{enumerate}
		\item[(a)] Power Spectrograms, which show the variation in signal power over time for a selected band of frequencies. These plots are sometimes also referred to as waterfall plots; % , which show ...; Need to add more info on plotting methodology
		\item[(b)] Min-max-mean Power Spectral Density plots with a threshold, which show the minimum, maximum and mean signal power over a selected band of frequencies. We take the help of these plots to draw our decision threshold that would be used to generate the duty-cycle plots (also known as occupancy charts); and% , which show ...; and
		\item[(c)] Occupancy Charts, which show the percentage of given time the RF signal was above the decision threshold. %, which show ... .
	\end{enumerate}
	By doing a band-by-band comparison we extract those bands where the variations in user occupancy are distinct for particular events (like a Virginia Tech football game). In section 4, we draw conclusions on the basis of the variations observed.

\subsection{Related Work}

In 1998, a large scale spectrum measurement campaign was conducted by the National Telecommunications and Information Administration (NTIA), a Federal agency that is responsible for regulating the use of government-held spectrum in the United States \cite{750342}. The work done by \textit{Wellens et. el} \cite{4549835}, describes a spectrum occupancy measurement campaign conducted in the city of Aachen, Germany. The campaign used a spectrum monitoring system making spectrum measurements from 20 MHz to 6 GHz. % They have used Amplitude Probability Distribution (APD), to investigate primary user activity.
Another major measurement campaign described in \textit{McHenry et. el} \cite{McHenry:2006:CSO:1234388.1234389} and conducted by Shared Spectrum Company (SSC) involved making spectrum occupancy measurements at several different locations including Northern Virginia, Chicago, Dublin in Ireland, New York City and National Radio Astronomy Observatory (NRAO) in West Virginia \cite{SSCSpecReportIreland} \cite{SSCSpecReportChicago} \cite{SSCSpecReportRiverbendPark} \cite{SSCSpecReportNYC} \cite{SSCSpecReportNRAO}. In these reports, SSC researchers have made occupancy measurements from 30 MHz to 2900 MHz and have plotted maximum power spectral density, power spectrogram, waterfall plots and duty-cycle occupancy charts. In these reports they have compared the occupancy per-band between different places. From these reports, we can see that average measured occupancy in rural areas is typically between 1\% and 3.4\% and in urban areas is typically between 11.4\% and 17.4\%. % A critical aspect of these measurements is the size of the data-set used for comparison varies over different places and as it is not coherent.
The surrounding events on the day of the measurement also impacts the measured occupancy at any place. Our work in this report follows a similar procedure, focusing on bands which show consistent variation pattern between Virginia Tech game days and regular college days. 

This study is a part of the Global RF Spectrum Opportunistic Assessment project, a WiFiUS initiative \cite{HAGER_GRANT}. Our partners in this project, the Illinois Institute of Technology, Chicago and VTT, the University of Oulu, Turku University of Applied Sciences, Finland have successfully installed similar spectrum observatories and have been carrying out measurements from almost a year. We sought help from our partners during the installation and configuration of our spectrum observatory. Reference \cite{6849666} describes the spectrum observatory installation and compares occupancy measurements between locations.

% (involving Chicago Spectrum Occupancy Measurements \& Analysis and a long term Studies Proposal, Mark McHenry's paper)

% Evaluation of Spectrum Occupancy in Spain for Cognitive Radio Applications

\section{Virginia Tech Spectrum Observatory System}

\subsection{RF Spectrum Monitoring System Installation}

\begin{figure}[h!]
  \centering
    \includegraphics[width=0.85\textwidth]{observatory_high_level_design.png}
	\caption{High Level System Design}
\end{figure}

The Virginia Tech Spectrum Observatory was installed during May 2014 and has been continuously monitoring RF spectrum from 30 MHz to 6 GHz since then. Figure 1 shows a high level system design of our spectrum observatory. The spectrum observatory has the following essential components:
\begin{enumerate}
	\item[a.] \textbf{RFEye Spectrum Sensor Node} - This core component is capable of acquiring real-time spectrum data and has the following hardware specifications:
	\begin{enumerate}
		\item[i.] The in-built receiver has a range of 10 MHz to 6 GHz (can be extended up to 18 GHz with a Block down converter).
		\item[ii.] It has a noise figure of typically 8 dB for 10 MHz - 4 GHz and 11 dB for 4 - 6 GHz respectively.
		\item[iii.] The receiver's spurious free dynamic range is at least 60 dB.
	\end{enumerate}
	More information on the RFEye receiver specification can be found at \cite{rfeye_specs}.
	Among the interfaces, we have, 
	\begin{enumerate}
		\item[i.] Four SMA inputs that can receive raw or filtered RF signals coming from the antenna output, 
		\item[ii.] Two USB  2.0 ports (currently unused) that can be used to connect to local storage devices like flash drives for back-up and collecting logs, 
		\item[iii.] GPS SMA interfaces, which connects the built-in GPS receiver to a GPS antenna. This helps in synchronizing the system time with the UTC time.
		\item[iv.] Cellular modem SMA interface (currently unused), which connects the built-in quad-band 850/900/1800/1900 MHz GSM/GPRS, UMTS/HSDPA transceiver to a compatible antenna. It has a built-in SIM card port too.
		\item[v.] A 1 Gbps Ethernet port is the only interface through which the node can be accessed and configured. The Ethernet port is also used to store collected spectrum data to an external network hard drive. We also provide power to the RFEye node using power-over-Ethernet (PoE).
	\end{enumerate}
	
	From a software perspective, the RFEye node runs a Linux based operating system with full C and Python development environments available. The system runs an in-built proprietary web server, as well. A web interface can then be used to configure and control the RFEye node remotely.
	Some of the essential web-based applications that are used to configure and collect data are:
	\begin{enumerate}
		\item[i.] NCPD, Control protocol server - This application is used to configure the RFEye node. It is used to set the node information, RF input information, and clock settings. The current NCPD configuration is provided in the appendix section.
		\item[ii.] GPSD, GPS application - This application is used to configure the parameters of the built-in GPS receiver, including the debug level, position state (fixed or moving), communication port, and baudrate.
		\item[iii.] LOGGER, logger application - This application is used to configure the collection and storage of spectrum data. Using a flexible configuration file, a detailed plan of spectrum observations can be defined. More on the configuration of the logger will be discussed in the subsection, ``Configuration of RFEye node''.
	\end{enumerate}
	
	\item[b.] \textbf{MP Antenna Super-M Ultra Base} - This is an ultra wideband spider antenna with a receive frequency range from 25 MHz to 6 GHz. The antenna is clamped to a seven foot mast and has a N-type jack which is connected to the input of the three-way splitter.
	\item[c.] \textbf{Three way splitter} - The signal received by the antenna goes to the input of the three way splitter. Of the three splitted outputs, one is filtered and one is filtered and amplified before being fed into inputs of the RFEye node. The third one is directly fed into an input of the RFEye node. 
	\item[d.] \textbf{300 MHz high pass filter} - One of the outputs of the three-way splitter is high pass filtered at 300 MHz to remove interference due to VHF signals.
	\item[e.] \textbf{3 GHz high pass filter \& high gain wideband amplifier} - One of the outputs from the three-way splitter is processed through a 3 GHz high pass filter which is in turn fed into a high gain wide band amplifier. The output signal from the amplifier is fed into an input of the RFEye node.
	\item[f.] \textbf{Network hard drive} - We use a 3 TB Western Digital network hard drive for remotely storing the spectrum data acquired by the RFEye node.
\end{enumerate}

\subsection{Configuration of RFEye node}
	
	The logger application accepts a configuration file which provides complete information on spectrum scans and the associated data storage locations. We have configured this file according to the band plan developed as part of our research project. We can configure each scan by setting the parameters like start frequency, stop frequency, sweep time, and resolution bandwidth. We can also configure the location of the log file, the location of the data directory (where the spectrum data will be stored), file size, and log level. Once the logger configuration file is prepared, it is fed through the web interface and the logger application is activated. With the help of web interface, we can check on the status of the logger application.

\subsection{Selecting an appropriate location}
	
	The observatory was installed on the roof-top of Whittemore hall. This location is less than a mile from Downtown Blacksburg and from Lane Stadium.
	
	\begin{figure}[ht!]
  \centering
    \includegraphics[width=0.85\textwidth]{observatory_site.png}
		\caption{Observatory installed on the roof of Whittemore Hall}
	\end{figure}

Figure 2 shows the observatory location in Google maps with respect to the downtown Virginia Tech and Lane Stadium.

	% Explanation about the band-plan
%%% ******************************** Complete till here ************************************************* %%%

\section{Spectrum Measurements and Occupancy comparison}

In this section we will use the collected measurement data from the spectrum observatory and carry out an occupancy comparison. Our first subsection will deal in finding bands which vary in their occupancy between Virginia Tech football game days, regular weekdays, and regular weekend days. To identify such bands, we examined spectrum activity from 30 MHz to 3 GHz.

In the second subsection we repeat the same exercise, comparing between Blacksburg and Chicago's spectrum data. The measurements in Chicago were collected using a similar spectrum observatory situated on the rooftop of the IIT tower.
We arbitrarily selected the same weekday, Aug 27, 2014 for Blacksburg and for Chicago. We then examined the spectrum from 30 MHz to 3 GHz to find bands which show a significant occupancy variation.
% The thresholds for the occupancy measurements were carefully selected by plotting the minimum, maximum and mean power spectral density.

	\subsection{RF Measurements and Occupancy for Virginia Tech Game Day}
	
	In this subsection we will be comparing the spectrum bands which show significant variation in their occupancy during Virginia Tech football game days as compared to a regular college day and a regular weekend day. We arbitrarily chose the first college football game of Virginia Tech's 2014 season, which was held on Aug 30, 2014. We also picked arbitrary weekday Aug 27, 2014 and a weekend day (Fall Break) on Oct 11, 2014, for comparison. % TODO Change the regular weekend day to Oct 18, 2014. 
	As we scan through the RF spectrum, we notice some variations in the following band:
	
		\subsubsection{Measurement comparison for the LMR band of 450 MHz to 470 MHz}
		
		From figure 3, we can see that Virginia Tech game day has a higher occupancy than the regular weekday. On October 11 % TODO Change it to October 18
		, the occupancy is much lower than a regular weekday or a game day, due to the fact that many students and others were out of the town for Fall Break, resulting in minimal usage of public safety systems. We use markers to point out frequency segments that show a higher occupancy on a game day. We further investigate the usage of specific channels using Blacksburg's LMR spectrum database \cite{blacksburg_spec_occ} which gives detailed information about the incumbent users of each.
		
		\begin{figure}[ht!]
			\centering
				\includegraphics[width=0.95\textwidth]{LMR_comparison_gameday.png}
			\caption{450 - 470 MHz LMR band game day occupancy comparison}
		\end{figure}
		
		\begin{enumerate}
			\item[(a)] Marker 1 shows a higher occupancy on certain frequencies. 451.8625 MHz is used by Virginia Tech Parking Services, 452.2 MHz is used by Virginia Tech for special events, and 453.0125 MHz is used by Virginia Tech University Unions and Student Activities. A game day is more busier than a regular day which in turn is busier than a regular weekend. There is also a surge in the occupancy of Montgomery County's Sheriff's dispatch which operates at 452.3, 452.35 and 453.2 MHz.
			\item[(b)] Marker 2 also shows a higher occupancy on certain frequencies. 461.7875 and 461.8875 MHz are used for Virginia Tech Emergency Medical Services (EMS) and facility services, respectively. Virginia Tech Athletics operates at 463.4875 and 464.5875 MHz.
		\end{enumerate}
		
		Thus we see a perceivable change in occupancy in those LMR channels which are directly associated with the event and also the services that are influenced by the event.
		
		Apart from browsing other bands for variations in game day occupancy, we expected that the cellular band would be a candidate for occupancy comparison. However after going through the measurement plots of cellular band's uplink, the occupancy numbers were misleading because these handheld devices transmit at low power causing an energy detector to miss a lot of uplink transmissions. For cellular downlink frequencies, we didn't find any variations in the power levels between a game day, a regular weekday, and a regular weekend day. 
 	
	\subsection{RF Measurements and Occupancy Compared Between Blacksburg, VA and Chicago, IL}
	
	In this subsection we examine the spectrum bands of both Chicago and Blacksburg which show significant differences in terms of occupancy. We also try to add information about the incumbent users operating in these bands and the wireless technology they are using.
	
		\subsubsection{Scanning through the Land Mobile Radio (LMR) band}
		
		\begin{figure}[ht!]
			\centering
				\includegraphics[width=0.95\textwidth]{LMR_comparison_Blacksburg_Chicago.png}
			\caption{Duty cycle plot from 450 - 470 MHz LMR band for Blacksburg and Chicago}
		\end{figure}
		
		Figure 4 shows the duty cycle plot comparing spectrum occupancy in Blacksburg and Chicago for August 27, 2014 (a weekday) for 450 - 470 MHz, part of the LMR band. For this LMR band comparison, we chose a threshold of -93 dBm for the Blacksburg data and -109 dBm for the Chicago measurements (see appendix for more information on decision thresholds). From the figure, we see that the bands in Chicago show a heavier occupancy than those in Blacksburg, as expected. Using the frequency database found in \cite{chicago_spec_occ} \cite{blacksburg_spec_occ}, we see that there are numerous services that are being offered in this band, especially in the city of Chicago. Thus, we have markers to explain those frequency segments that interests us.
		
		\begin{enumerate}
		
			\item[(a)] Marker 1 shows around 10\% occupancy in Blacksburg and around 2\% occupancy in Chicago for the channel occupying the frequency segments 450.8 - 451.4 MHz. In Blacksburg, the channels in 450.8 - 451.4 MHz segment having center frequencies 451.1, 451.275 and 451.375 MHz are used by the Town of Blacksburg's fire dispatch, tactical command (tac) and Montgomery County Sheriff, respectively. In Chicago, school services and public works telemetry services are offered at the frequencies 451.1875, 451.1125, 451.1375, 451.1625 and 460.175, with some of these frequencies being spatially reused within Chicago. 
			Thus, although there are services offered in Chicago using these frequencies, Figure 2 suggests that they are used less intensively than in Blacksburg, at least at the measurement location in Chicago.
			
			\item[(b)] Marker 2 shows comparatively higher occupancy for Chicago than Blacksburg. We compare the channels allocated and services provided within each frequency segment identified by these markers:
			
			\begin{enumerate}
				
				\item[(i)] In Chicago the channels in the 451.75 - 452.5 MHz segment are used mainly for trunked radio systems (which serve multiple services with the same channel or set of channels). For example, 451.7675 MHz is used for operations in schools located in Riverside and Brookfield, and 461.81250 MHz is used for security, operations, and maintenance of Proviso Township High School. The segment also provides channels that are allocated for private businesses. For example, 451.85 MHz is used for Cargo/Shipping companies and 452.125 MHz is used for Delta airlines ground operations. This segment also provides transportation services. For example, 452.3625 and 452.3875 MHz are used by taxi companies.
				
				In Blacksburg, the channels in the 451.75 - 452.5 MHz segment are used by Virginia Tech Parking services at 451.8625 MHz, Montgomery County's Fire tac which operates at 451.7875 MHz, and Virginia Tech's Communication Network Services operating at 452 MHz. This segment is also used by EMS services at 452.425 and 452.25 MHz, provided by the Riner Rescue Squad and Christiansburg Rescue Squad, respectively. Apart from this, Virginia Tech uses 452.2 MHz for special events (including football games).
				
				\item[(ii)] In Chicago, the channels in the 453 - 453.5 MHz segment are used for services like transportation and public works. For example, the Chicago Transit Authority uses 453.225, 453.375, 453.425 and 453.475 MHz to provide wireless communications for their buses. The parking garage services at Evanston use 453.0625 and 453.0875 MHz. The Metropolitan Water Reclamation District uses 453.275 and 453.3375 MHz.
				
				In Blacksburg, the channels in the 453 - 453.5 MHz segment are used for school services and law and order. For example, the Blacksburg Police Department's dispatch services are at 453.4875 MHz.
				
			\end{enumerate}
			
			\item[(c)] Marker 3 shows relatively higher occupancy for Blacksburg than Chicago in channels that are in the range of 453.6 - 454 MHz. In Blacksburg, Virginia Tech Police Department operates dispatch services at 453.875 MHz, and the Blacksburg Transit Authority uses 453.625 and 453.85 MHz for their operations. The Montgomery County Sheriff also offers law and order services in this frequency segment. In Chicago, there are few allocations in this segment, and they provide services mostly related to public works. For example, the ground maintenance services offered in North Berwyn Park District use 453.6375 MHz.
			
			%\item[(d)] Marker 4 shows heavy occupancy only in Chicago area and almost no occupancy in Blacksburg. The channel segments that fall in this category are 454 - 456 MHz.
			%
			%\begin{enumerate}
				%\item[(i)] The channel segment
				%\item[(ii)] The channel segment
			%\end{enumerate}
			
			\item[(d)] From marker 4 we can see that there is a higher occupancy in Blacksburg centered at 460 MHz. This is because of an unknown incumbent user centered at 460 MHz. In Chicago, its Police Department uses the 460 - 460.6 MHz band for their operations citywide. EMS teams also use this band for their operations in Chicago. Thus, we can see relatively a higher occupancy from 460 to 460.6 MHz band in Chicago than in Blacksburg.
			
	\end{enumerate}
	
	Looking at the overall occupancy, 4.06\% and 10.09\% in Blacksburg and Chicago, respectively, it is quite obvious that the city of Chicago has more channels allocated \cite{chicago_spec_occ} and a heavier occupancy as compared to Blacksburg. Also, some channel segments in this 450 - 470 MHz band have active trunked radio systems in order to improve the utilization of the band in those frequency segments. But, from markers 1 and 3, we can see that there are some segments where Blacksburg's channel occupancy is higher than Chicago's. This is because, in Blacksburg, these marked channel segments provide crucial services related to the law and order and transportation, while the same segments are used for lower-utilization public works in Chicago.
	
	Another band that has a higher spectrum allocations and occupancy in Chicago is the 700 MHz LMR band. The 700 MHz Public Safety band extends from 763 - 775 MHz and 793 - 805 MHz \cite{fcc_700_MHz_lmr}. This band is divided into three segments \cite{fcc_700_MHz_lmr}:
	
	\begin{enumerate}
		\item[(i)] 763-768/793-798 MHz segment allocated for public safety broadband communications
		\item[(ii)] 768-769/798-799 MHz segment is reserved as a guard band
		\item[(iii)] 769-775/799-805 MHz segment allocated for public safety narrowband communications - The two narrowband segments, 769 - 775 MHz are used for base operations and 799 - 805 MHz used for mobile operations. Each narrowband segment is divided into 960 channels, with each channel having a size of 6.25 kHz.
	\end{enumerate}
	
	\begin{table}[ht!]
	\centering
		\begin{tabular}{| p{1cm} | p{6cm} | p{4cm} |}
		\hline
		No. & Service Name & Frequencies used (in MHz)	\\	\hline
		1. & Cook County's Sheriff's Police Department & 769.5125, 773.1375, 774.9125, 769.2625, 769.8125, 770.2625, 771.2625, 771.5375, 771.9875, 772.3875, 773.3875, 773.6375 \\ \hline
		2. & Pace Suburban Bus Service of the RTA & 772.64375	\\	\hline
		3. & DuPage County's Emergency Telephone System Board (ETSB) & 769.33125, 769.75625, 770.05625, 770.80625	\\
		\hline
		\end{tabular}
		\caption{769 - 775 MHz LMR allocations in Chicago}
	\end{table}
	
	\begin{figure}[ht!]
		\centering
			\includegraphics[width=0.95\textwidth]{700_MHz_LMR_comparison_Blacksburg_Chicago.png}
		\caption{Duty cycle plot from 769 - 774 MHz LMR band for Blacksburg and Chicago}
	\end{figure}
	
	The narrowband segment at 769-775 MHz is of particular interest. As we can see in figure 5, this narrowband segment shows heavy occupancy in Chicago and almost no occupancy in Blacksburg. The primary incumbents who operate in these bands are the Cook County Sheriff, the Pace suburban bus service and DuPage County's Emergency Telephone System Board. Table 1 shows which entities are operating at particular frequencies.
	
	The markers in the figure 5 correspond to the numbers in Table 1. Thus we can see a high occupancy of the 700 MHz LMR spectrum in Chicago, while the band is not at all used in Blacksburg.
	
	\subsubsection{Comparison of occupancy in Television bands}
	
	\begin{table}[h!]
		\begin{tabular}{| p{2cm} | p{2cm} | p{3cm} | p{2cm} | p{2cm} | p{2cm} | p{2cm} |}
		\hline
		RF channel \#	& Frequency Range (In MHz)	& Callsign 	& Network & Location & Power (in kW) \\	\hline
		
		3 & 60 - 66 & WBRA-TV & PBS & Roanoke & 8 \\ \hline
		% VC 15 Occupied
		13 & 234 - 240 & WSET-TV &	ABC & Lynchburg	& 28.7 \\ \hline
		% VC 13 Not Occupied
		17 & 488 - 494 & WFXR & Fox & Roanoke & 695 \\ \hline
		% VC 27 Occupied
		18 & 494 - 500 & WDBJ & CBS &	Roanoke &	460 \\	\hline
		% VC 7 Occupied
		20 & 506 - 512 & WWCW & CW & Lynchburg	&	916 \\	\hline
		% VC 21 Not Occupied
		30 & 566 - 572 &	WSLS-TV &	NBC &	Roanoke & 1000 \\	\hline
		% VC 10 Occupied
		36 & 602 - 608 & WPXR-TV & ION & Roanoke & 700 \\
		% VC 38 Occupied
		\hline
		\end{tabular}
		\caption{Broadcast TV channels near Blacksburg, VA}
	\end{table}
	
	We next compare the spectrum allocation and occupancy of Broadcast Television (TV) band in Chicago and Blacksburg. Most TV signals that can be potentially received in Blacksburg are from Roanoke which is approximately 40 miles away, while some originate in Lynchburg, which is approximately 60 miles away. In Chicago, TV broadcast occurs from within city limits and nearby. Table 2 lists the TV channels that may possibly be received in Blacksburg \cite{tv_channel_info_blacksburg} \cite{tv_channel_frequencies}. Similarly, table 3 lists the TV channels that may possibly be received in Chicago \cite{tv_channel_info_chicago} \cite{tv_channel_frequencies}. These tables give us information about the primary incumbents of a particular channel. Using the power spectrogram plots, though, we compare the occupancy of TV bands in the two locations.
	
	\begin{table}[ht!]
		\begin{tabular}{| p{2cm} | p{2cm} | p{3cm} | p{2cm} | p{2cm} | p{2cm} | p{2cm} |}
		\hline
		RF channel \#	& Frequency Range (in MHz)	& Callsign 	& Primary User 	& Location 	& Power (in kW) \\	\hline
		
		12 & 204 - 210 & WBBM-TV & CBS & Chicago, IL & 8 \\	\hline
		% Occupied
		17 & 488 - 494 & WYIN & Northwest Indiana Public Broadcast & Gary, IN & 300 \\ \hline
		% Occupied
		19 & 500 - 506 & WGN-TV & CW & Chicago, IL & 645 \\	\hline
		% Occupied
		21 & 512 - 518 & WYCC & PBS & Chicago, IL & 98.9	\\	\hline
		% Occupied
		27 & 548 - 554 & WCIU-TV & Independent station & Chicago, IL & 160 \\	\hline
		% Occupied
		29 & 560 - 566 & WMAQ-TV & NBC & Chicago, IL & 350 \\	\hline
		% Occupied
		31 & 572 - 578 & WFLD & Fox & Chicago, IL & 1000	\\	\hline
		% Occupied
		36 & 602 - 608 & WJYS & Independent station & Hammond, IN & 50	\\	\hline
		% Occupied
		38 & 614 - 620 & WGBO-DT & UNI & Joliet, IL & 600	\\	\hline
		% Occupied
		43 & 644 - 650 & WCPX-TV & ION & Chicago, IL & 200 \\	\hline
		% Occupied
		44 & 650 - 656 & WLS-TV & ABC & Chicago, IL & 1000	\\	\hline
		% Occupied
		45 & 656 - 662 & WSNS-TV & TEL & Chicago, IL & 467 \\	\hline
		% Occupied
		47 & 668 - 674 & WTTW & PBS & Chicago, IL & 300 \\	\hline
		% Occupied
		50 & 686 - 692 & WXFT-DT & TF & Aurora, IL & 172	\\	\hline
		% Occupied
		51 & 692 - 698 & WPWR-TV & MNT & Gary, IN & 1000	\\
		% Occupied
		\hline
		\end{tabular}
		\caption{Broadcast TV channels near Chicago, IL}
	\end{table}
	
	\begin{figure}[ht!]
		\centering
			\includegraphics[width=0.95\textwidth]{470_512_TV_Band_Comparison.png}
		\caption{Power Spectrogram from 470 to 512 MHz for the entire day of Aug 27, 2014 comparing Chicago and Blacksburg}
	\end{figure}
	
	Figure 6 shows power spectrogram plots comparing Blacksburg and Chicago from 470 to 512 MHz. From the figure, we can see that for Blacksburg, channel 17 is occupied. From Table 2 we can see that RF channel 17 is operated by Fox. The TV transmitter for this station is located in Roanoke. We can see that the same channel is occupied in Chicago, operated by Northwest Indiana Public Broadcast company. The transmitter for that station is located in Gary, IN which is about 40 miles away from Chicago. 
	For Blacksburg, we can see that the RF channel 18 is also occupied (by CBS). In Chicago, the same channel is unoccupied. We can also see that RF channel 20 appears unoccupied in Blacksburg, though the broadcaster CW Television Network transmits on that channel from Lynchburg. Similarly, RF channel 13, also transmitting in Lynchburg, is not apparent on the Blacksburg spectrograph. 
	
	\begin{figure}[ht!]
		\centering
			\includegraphics[width=0.95\textwidth]{512_608_TV_Band_Comparison.png}
		\caption{Power Spectrogram from 512 to 608 MHz for the entire day of Aug 27, 2014 comparing Chicago and Blacksburg}
	\end{figure}
	
	\begin{figure}[ht!]
		\centering
			\includegraphics[width=0.95\textwidth]{608_698_TV_Band_Comparison.png}
		\caption{Power Spectrogram from 608 to 698 MHz for the entire day of Aug 27, 2014 comparing Chicago and Blacksburg}
	\end{figure}
	
	Figures 7 and 8 also show power spectrogram plots comparing Blacksburg and Chicago from 512 to 608 MHz and 608 to 698 MHz, respectively. While comparing both these figures, we find a good correspondance with the channels listed in Table 2 and 3, found occupied. The location of the transmitters for TV channels 50, 36 and 38 are all within 50 miles of Chicago.
	
	Also, from Figures 7 \& 8, RF channels 25, 26, 30, 32, 39 and 40 appear to be occupied in Chicago, but the primary users identity is not known.
	
	From the above observations we see that the number of channels occupied in Chicago is definitely higher than the number occupied in Blacksburg, which means more ``TV whitespace'' is available in Blacksburg than in Chicago.
	
	%NOTE: Distance of these cities from Chicago city:
	%
	%Gary, IN (40 miles from Chicago city)
	%Aurora, IL (41.5 miles from Chicago city)
	%Hammond, IN (25 miles from Chicago city)
	%Joliet, IL (40 miles from Chicago city)
	%LaSalle, IL (100 miles from Chicago city)
	
	%\subsubsection{Peeking into the highly congested cellular band}
	%
	%The cellular band includes that were studied in this measurement comparison exercise are:
	%\begin{enumerate}
		%\item[(i)] 700 MHz cellular band: The 700 MHz cellular band includes 700 - 750 MHz, 750 - 763 MHz, 775 - 793 MHz
		%\item[(ii)] PCS band: The PCS band includes bands from 1850 to 1910 MHz and from 1930 to 1990 MHz.
		%\item[(iii)] AWS band: The AWS band includes band
	%\end{enumerate}
	%We scanned through these bands by following the FCC spectrum dashboard \cite{fcc_dashboard} which provided a decent idea about the primary users. But, we couldn't get much information on what technology they are operating on what band. From \cite{lte_spectrum}, which says 

%@ELECTRONIC{lte_spectrum,
		%author = "Phil Goldstein",
		%title = "How much {LTE} spectrum do {V}erizon, {AT}\&{T}, {S}print and {T}-mobile have - and where?",
		%howpublished = "\url{http://www.fiercewireless.com/special-reports/how-much-lte-spectrum-do-verizon-att-sprint-and-t-mobile-have-and-where}"
%}
	
	%\subsubsection{Other bands}
	%
	%One such band that caught my interest while sweeping through the RF spectrum from 30 MHz to 3 GHz is the Paging and Radio location band. 
	

\section{Conclusion}

In this report, we pursued a broad-based RF spectrum survey of Blacksburg. The measurements for this survey were collected from the Virginia Tech spectrum observatory, installed on the roof-top of Whittemore Hall which is continuously acquiring and logging measurements onto a network hard drive. We use these measurement logs to plot minimum, maximum and mean power spectral densities which helped in selecting the decision threshold. We also plotted power spectrograms and duty cycle plots examining per-band occupancy, thereby selecting bands showing significant variation.

In this study, we found significant variation in occupancy for the 450 - 470 MHz LMR band during Virginia Tech football game weekend days compared to regular weekend days and regular weekdays. From this comparison, we conclude that the occupancy is distinctly higher in the channels that used to provide services which are directly related to the game event. 

We also examined bands showing significant variation between Blacksburg and Chicago. Due to the large number of services provided in Chicago's 450 - 470 MHz LMR band, particularly trunked radio services, we see a higher occupancy in most of the LMR channels compared to Blacksburg. 

In the 700 MHz LMR band, we also see a higher occupancy in Chicago as compared to Blacksburg, where the band is almost empty.

Lastly, we see a higher occupancy in the TV bands in Chicago than in Blacksburg.

\section*{Appendix}

\subsection*{Current NCPD configuration}

\begin{verbatim}

{
	"version":1,
	"information":
		{
			"node_name": "RFEYE00XXXX",
			"node_description": "Blacksburg Virginia Tech WiFi-US Spectral Analysis",
			"update_user": "WiFi-US Virginia Tech",
			"update_time": "06:35"
		},
	"settings":
		{
			"real_time_clock":"gps",
			"reference_clock":"gps",
			"gps_time_offset": -5
		},
	"antennas":
		[
			{
				"uid": 2,
				"name": "Input 2",
				"description": "SUPER M ULTRA BASE ANTENNA #1 - 3GHz HPF and amplified signal",
				"rf": 2,
				"range": {"min": 10, "max":6000}
			},
			{
				"uid": 3,
				"name": "Input 3",
				"description": "SUPER M ULTRA BASE ANTENNA #1 - raw signal from splitter",
				"rf": 3,
				"range": {"min": 10, "max":6000}
			},
			{
				"uid": 4,
				"name": "Input 4",
				"description": "SUPER M ULTRA BASE ANTENNA #1 - 300 MHz HPF signal",
				"rf": 4,
				"range": {"min": 10, "max":6000}
			}
		]
}

\end{verbatim}
\pagebreak

\subsection*{Decision Threshold selection in the 450 - 470 LMR band comparison for Blacksburg and Chicago}

\begin{figure}[ht!]
	\centering
		\includegraphics[width=0.95\textwidth]{450_470_Blacksburg_decision_threshold_PSD.png}
	\caption{450 - 470 MHz LMR band Min-Max-Mean PSD with a threshold in Blacksburg}
\end{figure}

\begin{figure}[ht!]
	\centering
		\includegraphics[width=0.95\textwidth]{450_470_Chicago_decision_threshold_PSD.png}
	\caption{450 - 470 MHz LMR band Min-Max-Mean PSD with a threshold in Chicago}
\end{figure}

\subsection*{Blacksburg Game Day Occupancy Comparison Miscellaneous Plots}

\begin{figure}[ht!]
	\centering
		\includegraphics[width=0.95\textwidth]{57_88.png}
	\caption{57 - 88 MHz band game day Maximum PSD, Power Spectrogram and Duty-cycle plot}
\end{figure}

\begin{figure}[ht!]
	\centering
		\includegraphics[width=0.95\textwidth]{57_88_occ_chart.png}
	\caption{57 - 88 MHz band game day Occupancy comparison}
\end{figure}

\begin{figure}[ht!]
	\centering
		\includegraphics[width=0.95\textwidth]{88_108.png}
	\caption{88 - 108 MHz band game day Maximum PSD, Power Spectrogram and Duty-cycle plot}
\end{figure}

\begin{figure}[ht!]
	\centering
		\includegraphics[width=0.95\textwidth]{88_108_occ_chart.png}
	\caption{88 - 108 MHz band game day Occupancy comparison}
\end{figure}

\begin{figure}[ht!]
	\centering
		\includegraphics[width=0.95\textwidth]{108_130.png}
	\caption{108 - 130 MHz band game day Maximum PSD, Power Spectrogram and Duty-cycle plot}
\end{figure}

\begin{figure}[ht!]
	\centering
		\includegraphics[width=0.95\textwidth]{108_130_occ_chart.png}
	\caption{108 - 130 MHz band game day Occupancy comparison}
\end{figure}

% subsection*{Blacksburg and Chicago Occupancy Comparison Miscellaneous Plots}


%\subsection{Power spectrogram of TV Bands ranging 57 - 406 MHz comparing Chicago and Blacksburg}
%
%
%
%\subsection{Power spectrogram of Cellular Bands comparing Chicago and Blacksburg}


%%%%%%%%%%%%%%%%%
%
% Include an EPS figure with this command:
%   \epsffile{filename.eps}
%

%%%%%%%%%%%%%%%%
%
% Do tables like this:

 %\begin{table}
 %\caption{The Graduate School wants captions above the tables.}
%\begin{center}
 %\begin{tabular}{ccc}
 %x & 1 & 2 \\ \hline
 %1 & 1 & 2 \\
 %2 & 2 & 4 \\ \hline
 %\end{tabular}
%\end{center}
 %\end{table}

%%%%%%%%%%%%%%%%%%%%%%%%%%%%%%%%

% If you are using BibTeX, uncomment the following:
% \thebibliography
%
% Otherwise, uncomment the following:
% \chapter*{Bibliography}

% \appendix

% In LaTeX, each appendix is a "chapter"
% \chapter{Program Source}

\bibliographystyle{IEEETran}
\bibliography{reference}

\end{document}